\documentclass[dvipdfmx,uplatex,a4paper]{article}
\def\vector#1{\mbox{\boldmath $#1$}}
\usepackage[hiresbb]{graphicx}
\usepackage{amsmath}
\usepackage{amssymb}
\usepackage{ascmac}
\usepackage{braket}
\usepackage{circuitikz}
\usepackage{cite}
\usepackage{comment}
\usepackage{float}
\usepackage{listings}
\usepackage{nameref}
\usepackage{siunitx}
\usepackage{tikz}
\usepackage[version=3]{mhchem}

\usepackage[colorlinks=true, bookmarks=true,
bookmarksnumbered=true, bookmarkstype=toc, linkcolor=blue,
urlcolor=blue, citecolor=blue]{hyperref}

\makeatletter
 \renewcommand{\theequation}{
   \thesubsection.\arabic{equation}}
  \@addtoreset{equation}{section}
\makeatother

\title{Solve RGB+ LEDs PWM from Chromaticity}
\author{CHINZEI, Kiyoyuki}
\date{\today}
\begin{document}
\maketitle

\begin{abstract}
This article\footnote{To cite, please refer to https://github.com/kchinzei/kch-rgbw-lib/docs/rgbw\_solver.pdf} is a private note to develop \texttt{kch-rgbw-lib}, a TypeScript library in \href{https://github.com/kchinzei/kch-rgbw-lib}{github} and \href{https://www.npmjs.com/package/kch-rgbw-lib}{npm}. \texttt{Kch-rgbw-lib} provides classes and functions for multicolor LED, including color space conversions between HSV, RGB, XYZ and xyY and calculation of accurate color composition. This article gives a general solution of multi-number (more than R-G-B) LEDs to represent composite colors. It is not intended to carry new, comprehensive, or most efficient ideas. This document is granted under MIT License.
\end{abstract}

\section{Define the Problem}
\subsection{Past works}\label{s_intro}
Obtaining accurate colors by mixing R-G-B color sources has been utilized as color displays since 1950s. As various colors are available by LED, recent topics are solutions of color composite with additional color sources typically for OLED applications \cite{Chi2011, Lee2014}. Usually white light sources are used as an additional light source \cite{AN1562, Chi2011, Lee2014}. For display purposes additional colors other than R-G-B have been also used to expand the possible color ranges (gamut)\cite{Wikipedia_multicolor}. Sharp once added yellow in Aquos flat-panel TV, but they initially researched 5-primary color display \cite{Sharp2011}. Amber \cite{AN2026}, turquoise, and violet can be other colors to expand the gamut.

\subsection{Given parameters and assumptions}\label{s_assumptions}
We have $N \geq 3$ color sources (LEDs) with chromaticity $(x_i, y_i)$ and maximum luminance $Y_i$, where $i=1 \ldots N$. Our problem is to find the optimum composite output ratio (PWM output) $\boldsymbol{\alpha} = [\alpha_1 \ldots \alpha_N]^T$ where $0 \leq \alpha_i \leq 1$ to represent a given color input with chromaticity $(x, y)$ and luminance $Y$.

Here, we set our goal of optimization as the following:
\begin{enumerate}
  \item\label{I_max_luminance} Under physical constraint $0 \leq Y \leq  Y_1 + \ldots + Y_N$,
  \item\label{I_solution} Obtain exact composite color to represent $(x, y, Y)$.
  \item\label{I_inside_gamut} If $(x, y)$ is outside the gamut of color source, the nearest color in it is used.
  It can be achieved by projecting the input to the gamut contour.
  \item\label{I_min_energy} When possible, minimize energy consumption $E$, \\
  \begin{equation}
    \label{E_min_energy}
    E = \sum_1^N \alpha_i W_i
  \end{equation}
  where $W_i$ is the power of each LED at the maximum luminance.
  \item\label{I_lifetime} When possible, average turn-on time of LEDs to equalize LEDs lifetime,
  \item\label{I_small_alpha} When possible, set $\alpha_i$ to null when it's very small.
  It is preferable to avoid jitter at low PWM output.
\end{enumerate}

It is a typical linear programming (LP) problem. When $N=3$, e.g. R-G-B color sources only, it's a deterministic and not an optimization problem. And when $N=4$, e.g. R-G-B-W LEDs, there is only one parameter to optimize, which makes the problem as simple as we don't need to use sophisticated LP solver.
% Mathematically, it is dependent to $R = rank(\boldsymbol{A})$, where $\boldsymbol{A}$ is a matrix of LEDs, in Eq.~\eqref{E_X=AY}. We first derive a general description of the problem and solve it for $R=3$, $R=4$ and $R > 4$ cases.
We first derive a general description of the problem and solve it for $N=3$, $N=4$ and $N > 4$ cases.

\section{General solution}
We use XYZ color space because a composite of color source in XYZ color space can be obtained as a simple sum of each term.  In XYZ color space, our problem is to determine $\boldsymbol{\alpha} = [\alpha_1 \ldots \alpha_N]$ which gives an equation between input color $[X, Y, Z]^T$ and color source $[X_{i}, Y_{i}, Z_{i}]^T$, $i=1 \ldots N$;

\begin{equation}
  \label{E_XYZcomposite}
  \left[
    \begin{array}{c}
      X \\
      Y \\
      Z
    \end{array}
  \right]
   = \alpha_1
  \left[
    \begin{array}{c}
        X_1 \\
        Y_1 \\
        Z_1
    \end{array}
  \right]
   + \ldots + \alpha_N
  \left[
    \begin{array}{c}
        X_N \\
        Y_N \\
        Z_N
    \end{array}
  \right]
\end{equation}

Color space $[X, Y, Z]^T$ is expressed by using $(x, y, Y)$:

\begin{eqnarray}
  \label{E_xyY2XYZ}
  X & = & \frac{x}{y}  Y \\
  Y & = & Y \\
  Z & = & \frac{1 - x - y}{y} Y
\end{eqnarray}

Using matrix representation, Eq.~\eqref{E_XYZcomposite} is written as

\begin{equation}
  \label{E_X=AY}
  \left[ \boldsymbol{X} \right] =
  \left[ \boldsymbol{A} \right]
  \left[ \boldsymbol{\alpha} \right]
\end{equation}

\noindent
where

\begin{eqnarray}
  \left[ \boldsymbol{X} \right] &=&
  \left[ X, Y, Z \right]^T \label{E_XYZ^T} \\
%
  \left[ \boldsymbol{A} \right] &=&
  \left[
    \begin{array}{ccc}
      \frac{x_1}{y_1} Y_1 & \ldots & \frac{x_N}{y_N} Y_N \\
      Y_1 & \ldots & Y_N \\
      \frac{1 - x_1 - y_1}{y_1}Y_1 & \ldots & \frac{1 - x_N - y_N}{y_N}Y_N
    \end{array}
  \right] \\
%
  \left[ \boldsymbol{\alpha} \right] &=&
  \left[ \alpha_1,\ \ldots\ , \alpha_N \right]^T
\end{eqnarray}

Our goal is to solve Eq.~\eqref{E_X=AY} with respect to $ \boldsymbol{\alpha} = [\alpha_1 \ldots \alpha_N]^T $. To solve it, we can use the singular value decomposition (SVD) (Eq.~\eqref{E_SVD})~\cite{SVD_NRC}.

\begin{equation}
  \label{E_SVD}
  \Bigg[ \boldsymbol{A} \Bigg] =
  \Bigg[ \boldsymbol{U} \Bigg]
  \left[
    \begin{array}{cccc}
      \omega_1 & \\
        & \omega_2 &   &  0 \ldots 0 \\
        &  & \omega_3 &
    \end{array}
  \right]
  \Bigg[ \boldsymbol{V}^T \Bigg]
\end{equation}

\noindent
where $ \boldsymbol{A} $ is $ 3 \times N $,
$ \boldsymbol{U} $ is $ 3 \times 3 $,
$ [\omega_1 \ddots \omega_3, 0 \ldots 0] $ is $ 3 \times N $,
$ \boldsymbol{V}^T $ is $ N \times N $ matrixes
\footnote{Many implementations of SVD return 'economy' $\boldsymbol{V}$ in $ N \times 3$ instead of calculating full $N \times N$ size. Matlab, Octave without 'econ' option and \href{https://www.npmjs.com/package/svd-js}{svd-js} javascript package in NPM with 'f' option (This is my contribution!) give full $\boldsymbol{V}$.}
in this specific case, since $ \boldsymbol{A} $ is a $ 3 \times N $ matrix and there are upto 3 $\omega$'s. When $ N \geq 4 $, $ [\omega_1 \ddots \omega_3 ] $ is null-padded in $ 3 \times (N-3) $.
The pseudo-inverse matrix $ \boldsymbol{A}^{-1} $ is obtained by

\begin{equation}
  \Bigg[ \boldsymbol{A} \Bigg]^{-1} =
  \Bigg[ \boldsymbol{V}_{1-3} \Bigg]
  \left[
    \begin{array}{ccc}
      1/\omega_1 & & \\
      & 1/\omega_2 & \\
      & & 1/\omega_3
    \end{array}
  \right]
  \Bigg[ \boldsymbol{U}^T \Bigg]
\end{equation}

\noindent
where $ \boldsymbol{V}_{1-3} $ is the first 3 columns of $ \boldsymbol{V} $, those correspond to $ \omega_1 \ldots \omega_3 $
\footnote{Many implementations of SVD do not sort $ \boldsymbol{U}, \boldsymbol{V}$ by $\omega$s. Matlab and Octave do sort. Algorithm in 'Numerical Recipes in C' and \href{https://www.npmjs.com/package/svd-js}{svd-js} in NPM do not.}.
Using $ \boldsymbol{A}^{-1} $, we obtain

\begin{equation}
  \label{E_Y=A-1X}
  \left[ \boldsymbol{\alpha} \right] =
  \left[ \boldsymbol{A} \right]^{-1}
  \left[ \boldsymbol{X} \right]
\end{equation}

By the way, what about the rest of columns in $ \boldsymbol{V} $? They are null vectors of $ \boldsymbol{A} $. A null vector $\boldsymbol{n}$ of $ \boldsymbol{A}$ is such vector that satisfies $ \boldsymbol{A} \boldsymbol{n} = [0]$. By denoting these columns as $ \boldsymbol{n}_4 \ldots \boldsymbol{n}_N $, Eq.~\eqref{E_Y=A-1X} can be extended as

\begin{equation}
  \label{E_Y=A-1X+n}
  \left[ \boldsymbol{\alpha} \right] =
  \left[ \boldsymbol{A} \right]^{-1}
  \left[ \boldsymbol{X} \right]
  + \beta_4 \boldsymbol{n}_4 + \ldots + \beta_N \boldsymbol{n}_N
\end{equation}
\noindent
where $ \beta_4 \ldots \beta_N $ are arbitrary numbers. By choosing these using other constraints, we can obtain the optimum solution.

% The actual number of null vectors is determined by the rank of $\boldsymbol{A}$.




\section{Solution of $N=3$ case}

When $N=3$, Eq.~\eqref{E_Y=A-1X} gives a deterministic solution. No optimization. However, the obtained $ \alpha_i $ should be physically meaningful, i.e., $ 0 \leq \alpha_i \leq 1 $. It can be out of range when
\begin{itemize}
  \item Input color $[X]$ is out of the gamut defined by the color sources,
  \item $Y$ of $[X]$ is greater (brighter) than the color sources.
\end{itemize}
\noindent
When $ \alpha_i < 0$, the nearest color in the gamut should be used as the input. We can also truncate such $\alpha_i$ to 0. In this case, the output color has certain error.

When $ \alpha_i > 1 $, all $\alpha$s should be normalized by the largest $\alpha$. This way the color will be correctly obtained, but it will be darker than expected.




\section{Solution of $N=4$ case}

When $N=4$, Eq.~\eqref{E_Y=A-1X+n} is simple.

\begin{equation}
  \label{E_Y=A-1X+1}
  \left[ \boldsymbol{\alpha} \right] =
  \left[ \boldsymbol{A} \right]^{-1}
  \left[ \boldsymbol{X} \right]
  + \beta_4 \boldsymbol{n}_4
\end{equation}

\noindent
We will determine parameter $\beta_4$ using the assumptions in section \ref{s_assumptions}.
Since it is always $ 0 \leq \alpha_i \leq 1$, by solving it for $\beta$ (hereafter omitting '$_4$'), we obtain the following conditions.
\begin{eqnarray}
  \beta & \geq & \begin{cases}
    - \frac{b_i}{n_i} \;\;\;\; \text{ if \(n_i > 0\)} \label{E_betamin} \\
    \frac{1 - b_i}{n_i} \;\;\;\; \text{ if \(n_i < 0\)}
  \end{cases} \\
%
  \beta & \leq & \begin{cases}
    \frac{1 - b_i}{n_i} \;\;\;\; \text{ if \(n_i > 0\)} \\
    - \frac{b_i}{n_i} \;\;\;\; \text{ if \(n_i < 0\)} \label{E_betamax}
  \end{cases} \\
%
    E &=& \sum_{i=1}^4 (\beta n_i + b_i) W_i \to min \label{E_energy_n4}
\end{eqnarray}

\noindent
where $ [b_1,\, \ldots\, , b_4]^T = [ \boldsymbol{A} ]^{-1} [\boldsymbol{X}] $, $n_i$ are the elements of $\boldsymbol{n}$. Eq.~\eqref{E_energy_n4} is from Eq.~\eqref{E_min_energy}. Finding the largest and smallest values of the right hand side of Eqs.~\eqref{E_betamin} and \eqref{E_betamax}, denoted as $\beta_{min}$ and $\beta_{max}$, Eqs.~\eqref{E_betamin} and \eqref{E_betamax} are rewritten as

\begin{equation}
  \beta_{min} \leq \beta \leq \beta_{max}
\end{equation}

Since Eq.~\eqref{E_energy_n4} is rewritten as $E = s_1\beta + s_2$, here $s_1$ and $s_2$ are constants determined by calculating the sums in Eq.~\eqref{E_energy_n4}, optimized $\beta$ is determined as

\begin{equation}
    \beta = \begin{cases}
      \beta_{min}  & \text{if \(s_1 = \sum_1^4 n_i W_i > 0\)} \\
      \beta_{max}  & \text{else}
  \end{cases}
\end{equation}

\subsection{When $\beta$ is not determined}

When $\beta_{min} > \beta_{max}$, there is no feasible answer. Again, there are two cases, when the input color $[\boldsymbol{X}]$ is out of the gamut, or when $Y$ in Eq.~\eqref{E_XYZ^T} is greater than the color sources. Since $\alpha_i \geq 0$ cannot compromise, we find $\beta$ that satisfies

\begin{eqnarray}
  \beta & \geq & - \frac{b_i}{n_i} \;\;\;\; \text{ if \(n_i > 0\)} \label{E_beta0min} \\
  \beta & \leq & - \frac{b_i}{n_i} \;\;\;\; \text{ if \(n_i < 0\)} \label{E_beta0miax}
\end{eqnarray}
Then we normalize the largest $\alpha_i$ to be 1.

\subsection{When $\alpha_i$ is small}

To implement the assumption \ref{I_small_alpha} in section \ref{s_assumptions}, we can introduce allowance of small $\alpha$, $\alpha_\varepsilon$, and exchange Eqs.~\eqref{E_betamin} and \eqref{E_betamax} as

\begin{equation}
  \beta \geq \frac{\alpha_\varepsilon - b_i}{n_i}
\end{equation}

But this also needs to assert if $\beta_{min} \leq \beta_{max}$.




\section{Solution of $N > 4$ case}
We can optimize Eq.~\eqref{E_min_energy} under constraints of $0 \leq \alpha_i \leq 1$ for Eq.~\eqref{E_Y=A-1X+n} using a linear programming solution\cite{LP_NRC}.
To use linear programming we need to rewrite our problem into the \textit{normal form}. Normal form of a linear programming problem is

\begin{itemize}
  \item Objective functions are to be minimized,
  \item All constraints are equality formulas,
  \item All variables $ \geq 0$.
\end{itemize}

Since our constraints $0 \leq \alpha_i \leq 1$ are unequal and $\beta_i$ can be negative, we introduce \textit{slack variables} $\gamma_i, \delta_i$ and $\epsilon_i, \zeta_i$ to replace $\beta$.

\begin{itemize}
  \item Introduce $\gamma_i \geq 0$ such that $b_i + \beta_4 n_{4i} + \ldots + \beta_N n_{Ni} - \gamma_i = 0$,

  \item Introduce $\delta_i \geq 0$ such that $b_i + \beta_4 n_{4i} + \ldots + \beta_n n_{Ni} + \delta_i = 1$,

  \item Replace $\beta_i = \epsilon_i - \zeta_i$, such that $\epsilon_i \geq 0 $ and $\zeta_i \geq 0$.
\end{itemize}

After this modification, we have $2N$ equations with $2N$ variables (for $\gamma$ and $\delta$) and $2(N-3)$ variables (for $\epsilon$ and $\zeta$). Using these replacements, the objective function (Eq.~\eqref{E_min_energy}) and Eq.~\eqref{E_Y=A-1X+n} are rewritten as

\begin{eqnarray}
%
  \label{E_min_energy5}
  \sum_{i=1}^N (n_{4i} \epsilon_4 - n_{4i} \zeta_4 + \ldots +  n_{Ni} \epsilon_N - n_{Ni} \zeta_N + b_{i}) W_i &\to& min \\
%
  \label{E_bmin5}
  n_{4i} \epsilon_4  - n_{4i} \zeta_4 + \ldots + n_{Ni} \epsilon_N  - n_{Ni} \zeta_N  - \gamma_i + b_i &=& 0 \\
%
  \label{E_bmax5}
  n_{4i} \epsilon_4  - n_{4i} \zeta_4 + \ldots + n_{Ni} \epsilon_N  - n_{Ni} \zeta_N  + \delta_i + b_i &=& 1
\end{eqnarray}

\noindent
Eq.~\eqref{E_min_energy5} is the objective function. Solvers applicable to Eqs.~\eqref{E_min_energy5} - \eqref{E_bmax5} can be found in many numerical packages. \href{https://www.npmjs.com/package/kch-rgbw-lib}{\texttt{Kch-rgbw-lib}} uses \href{https://www.npmjs.com/package/linear-program-parser}{linear-program-parser} and \href{https://www.npmjs.com/package/linear-program-solver}{linear-program-solver} in npm.

\subsection{When solution infeasible}
Linear programming solver may find it's `infeasible' - no solution to satisfy all constraints in Eqs.~\eqref{E_bmin5} and \eqref{E_bmax5}. Another possibility is that the solution would not converge. Yet again, there are two cases, when the input color $[X]$ is out of the gamut, or when $Y$ in Eq.~\eqref{E_XYZ^T} is greater (brighter) than the color sources.

When the object brightness is too large, we can find a solution by omitting Eq.~\eqref{E_bmax5}. Physically doable solution is to normalize the maximum $Y_i$ to $1.0$.

We may also see infeasible case when one or more value in solution is very close to zero and Eq.~\eqref{E_bmin5} fails due to numerical error. If it is the case, you can set a small negative value to the right side of Eq.~\eqref{E_bmin5} instead of $0$.



\section{More to do}

The following sections discuss items not implemented in \href{https://www.npmjs.com/package/kch-rgbw-lib}{\texttt{kch-rgbw-lib}}. We leave them as `homework' fun.

\subsection{Not to compromise brightness $Y$}
When $Y$ in Eq.~\eqref{E_XYZ^T} is greater (brighter) than the color source's capacity, there is no solution. Here we suggested a (compromised) solution to normalize $Y_i$. But we can also take another strategy to maintain the goal brightness by compromising the goal color.

This strategy requires computation of gradient of $Y$ in $\boldsymbol{\alpha}$ space.


\subsection{LED's lifetime averaging}
In section \ref{s_assumptions} '\nameref{s_assumptions}' we introduced an optional condition \ref{I_lifetime}, 'When possible, average turn-on time of LEDs to equalize LEDs lifetime'. It is equivalent to minimize the variance of $\alpha_1 \ldots \alpha_N$. It can be an optimization of a quadratic function.

In $N = 4$ case, we have only one variable $\beta$ therefore we can optimize only one condition. This means we need to chose between the energy minimization and this condition.



\bibliographystyle{unsrt}
\bibliography{rgbw}

\section*{MIT License}

Copyright (c) 2020 Kiyo Chinzei

Permission is hereby granted, free of charge, to any person obtaining a copy
of this software and associated documentation files (the "Software"), to deal
in the Software without restriction, including without limitation the rights
to use, copy, modify, merge, publish, distribute, sublicense, and/or sell
copies of the Software, and to permit persons to whom the Software is
furnished to do so, subject to the following conditions:

The above copyright notice and this permission notice shall be included in all
copies or substantial portions of the Software.

THE SOFTWARE IS PROVIDED "AS IS", WITHOUT WARRANTY OF ANY KIND, EXPRESS OR
IMPLIED, INCLUDING BUT NOT LIMITED TO THE WARRANTIES OF MERCHANTABILITY,
FITNESS FOR A PARTICULAR PURPOSE AND NONINFRINGEMENT. IN NO EVENT SHALL THE
AUTHORS OR COPYRIGHT HOLDERS BE LIABLE FOR ANY CLAIM, DAMAGES OR OTHER
LIABILITY, WHETHER IN AN ACTION OF CONTRACT, TORT OR OTHERWISE, ARISING FROM,
OUT OF OR IN CONNECTION WITH THE SOFTWARE OR THE USE OR OTHER DEALINGS IN THE
SOFTWARE.

\end{document}
